\documentclass[a4paper, 11pt]{report}

\addtolength{\hoffset}{-1cm}
\addtolength{\textwidth}{2cm}

\usepackage[utf8]{inputenc}
\usepackage[frenchb]{babel}
\usepackage[T1]{fontenc}

\usepackage{tikz}


\usepackage{listings}
\usepackage{color}
\definecolor{lightgray}{rgb}{.9,.9,.9}
\definecolor{darkgray}{rgb}{.4,.4,.4}
\definecolor{purple}{rgb}{0.65, 0.12, 0.82}

\lstnewenvironment{OCaml}
                  {\lstset{
                      language=[Objective]Caml,
                      breaklines=true,
                      commentstyle=\color{purple},
                      stringstyle=\color{red},
                      identifierstyle=\ttfamily,
                      keywordstyle=\color{blue},
                      basicstyle=\footnotesize
                    }
                  }
                  {}
\lstnewenvironment{C}
                  {\lstset{
                      language=C,
                      breaklines=true,
                      commentstyle=\color{purple},
                      stringstyle=\color{red},
                      identifierstyle=\ttfamily,
                      keywordstyle=\color{blue},
                      basicstyle=\footnotesize
                    }
                  }
                  {}

\begin{document}

\chapter{Serveur OCaml - Béatrice Carré}

Le serveur a été réalisé en OCaml. Il utilise les modules Thread, Mutex
Unix, Arg (pour parser les options du serveur) et Str (pour les
expressions régulières du module Protocol), ainsi que ocamllex et
ocamlyacc pour la gestion des commandes.

Nous détaillerons par la suite les choix d'implémentation qui ont été faits.

\section{Etablissement du serveur}

L'établissement du serveur a été fait en s'inspirant du chapitre 20 de
\emph{Développement d'Applications avec Objective Caml}.
Comme spécifié dans le sujet du projet, la connexion s'appuie sur un
protocole TCP sur le port 2013 par défaut.
Le serveur a des paramètres par défaut modifiables par option au
moment du lancement :
\begin{itemize}
\item le \emph{timeout}, 30 secondes par défaut
\item le nombre de joueurs \emph{max} pour le lancement de la partie, 4 par défaut
\item le \emph{port} utilisé, 2013 par défaut
\item le nom du fichier \emph{dico}, ''dico.txt'' par défaut
\item le nombre \emph{n} de joueur(s) devant dénoncer un dessinateur pour
  qu'il soit considéré comme tricheur, 2 par défaut
\end{itemize}
Avant l'établissement du serveur, les arguments sont parsés
grâce au module \emph{Arg}.

\section{Traitement d'une commande - camllex et camlyacc}

Afin de traiter les commandes reçues, en respectant le protocole, il
était intéressant d'utiliser les générateurs d'analyseur syntaxique
ocamllex et de parser ocamlyacc. Le site \emph{http://caml.inria.fr/pub/docs/manual-ocaml-4.00/manual026.html}

Après avoir créé des constructeurs de commande dans le fichier \emph{protocol.ml}, il a
fallu définir la grammaire du parser (dans \emph{parser.mly}) et les
expressions régulières manipulées (dans \emph{lexer.mll}). La
grammaire traduit les expressions définies en commande en utilisant
les constructeurs du module Protocol.

Ainsi, commande et arguments sont aisément identifiés et manipulés. 
Une fois cela fait, pour traiter une commande reçue, il suffit
de combiner le lexer et le parser, pour
obtenir le type \emph{command} défini dans le protocole. Pour
envoyer une commande, il suffira d'utiliser les constructeurs définis
dans le module Protocol et de transformer la commande obtenue en chaîne de
caractère, grâce à la fonction string\_of\_command du fichier protocol.ml  

Le traitement des commandes se reposera par la suite sur le type somme
défini et les fonctions associées à sa manipulation.

\section{Connexion}

La phase de connexion est séparée de la boucle principale de jeu afin
d'identifier les conditions de connexion (joueur ou spectateur) et
lancer le traitement adéquat. Si le pseudo reçu est déjà utilisé
(joueur présent), alors un nouveau pseudo est généré en rajoutant un
entier entre parenthèse à la fin. Au bout du max-ième joueur se
connectant, la partie est lancée, dans un autre thread.

\section{Structures de données}

Lors de la connexion, la structure représentant le \emph{joueur} est
alors initialisée.

\begin{OCaml}
type player = {
  chan : Unix.file_descr; (*la socket par laquelle le serveur communique*)
  thread : Thread.t;
  name : string;
  mutable role : role; 
  mutable already_draw : bool; (*s'il a deja dessine pendant la partie *)
  mutable has_found : bool; (* s'il a deja trouve le mot*)
  mutable score_round : int; 
}
type role =
  | Undefined
  | Drawer
  | Guesser 

\end{OCaml}

Le role du joueur est représenté par un type somme qui nous permettra
aisément de vérifier si une commande est licite à un moment de la partie
donné selon son rôle.

Les joueurs sont stockés dans une liste (de taille maximale max),
ce qui permet d'utiliser les nombreuses fonctions du module List. 
Les spectateurs, stockés dans une autre liste, ne sont représentés que par leur socket.

Ces listes sont dans une structure représentant l'état de la
partie du \emph{server} :

\begin{OCaml}
type server = {
  mutable players : player list;
  mutable spectators : Unix.file_descr list;
  mutable mots_rounds : string list; (*les mots deja sortis pour
  eviter de tomber sur les meme mots*)
  mutable is_game_started : bool;
  mutable commandes : Protocol.command list (* les commandes envoyees
  depuis le debut de la partie*)
}


\end{OCaml}

Le \emph{round} est représenté par la structure présentée
ci-dessous :

\begin{OCaml}
type round = {
  timer : timer;
  mutable drawer : player option;
  mutable winner : string option;(*nom du premier joueur ayant trouve*)
  mutable word_to_find : string;
  mutable cpt_found : int; (* nombre de joueur ayant trouve le mot*)
  mutable nb_cheat_report : int;
  mutable color : color; (* par defaut noir*)
  mutable size : int; (* par defaut 0*)
}
\end{OCaml}

Et enfin, le \emph{timer}, qui a été implémenté en objet, pour pouvoir le 
manipuler facilement avec un accès simple aux méthodes.
Ce timer est créé avec un temps initial (le temps de la partie) et une
fonction de callback (ici, next\_round). Il comprend :
\begin{OCaml}
class timer init_delay callback = 
object(self)
  val mutable time = init_delay 
  val mutable running = false
  val mutex_delay = Mutex.create ()
  val mutex_running = Mutex.create ()
  method start_count ();
  method restart_count ();
  method get_current_delay;
  method set_delay new_delay;
  method stop_timer ();
end
\end{OCaml}

\section{Extensions réalisées}

\subsection{Discussion instantanée}

Cette extension a été plutôt simple à implémenter, dès que le serveur
reçoit la commande \emph{Talk message} correctement formée, il renvoie
à tout le monde (clients et spectateurs) le message passé en argument
précédé de son émetteur. 

\subsection{Courbes de Bézier}

Du côté serveur, cette extension est de la même manière rapide à
implémenter.
Il a juste fallu ajouter une commande au protocole fourni pour
permettre au dessinateur de dessiner une courbe : ``SET\_COURBE/x1/y1/x2/y2/x3/y3/x4/y4/''. 

\subsection{Spectateurs}

Pour cette extension, j'ai supposé que les seules commandes à envoyer
au spectateur est celle envoyées à tous les joueurs. Il a donc fallu
ajouter l'envoi des commandes aux spectateurs présents lorsqu'on fait
appel à la fonction \emph{broadcast} et de stocker la commande pour
permettre l'envoie de toutes les commandes précédemment envoyées
depuis le début de la partie, à un spectateur se connectant en pleine
partie



\section{Architecture}

L'architecture n'est pas forcément très rigoureuse car elle n'a pas
été développée dans un esprit de réutilisabilité. Lors de la connexion
d'une nouvelle socket, un thread exécutant \emph{init\_new\_client}
est lancé. Cette fonction lance la phase de connexion évoquée plus
haut, et selon le résultat exécute \emph{start\_player} ou
\emph{start\_spectator}. La deuxième fonction se contente de récupérer 
les commandes déjà effectuées et d'ajouter sa socket à la structure du
server. La première effectue une boucle qui match les commandes reçues
afin de lancer le traitement adapté selon son rôle.

\section{Choix d'implémentation}
L'énoncé étant parfois libre de choix, voici les décisions prises les
concernant :
 \begin{itemize}
\item Pour la commande Pass, j'ai respecté ce qui est décrit dans
  l'énoncé : cela produit un effet que si aucun joueur n'a deja trouvé
  le mot. Donc si un joueur a trouvé le mot, Pass ne marche pas du tout.
\item fin partie TODO L'enchaînement de plusieurs parties n'est pas géré.
\item La connexion d'un joueur en cours de partie n'est pas autorisé,
  pour éviter qu'un joueur ne commance la partie en ayant du retard au
  niveau des points. 
\item Les possibilités du spectators ne sont pas bien précisé, j'ai
  donc supposé qu'après se connexions, il ne pouvait que quitter le
  jeu avec la commande Exit. Cette commande prenant un nom en
  argument, elle est acceptée avec n'importe quel nom.
\end{itemize}

\end{document}
